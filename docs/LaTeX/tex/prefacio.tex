\chapter*{}

\thispagestyle{empty}

%%%%%%%%%%%%%%%%%%%%%%%%%%%%%%%%%%%%%%%%%%%%%%%%%%%%%%%%%%%%%%%%%%%%%%%%%%%%%%%%%%%%%%%
%%%%%%%%%%%%%%%%%%%%%%%%%%%%%%%%%%%%%%%%%%%%%%%%%%%%%%%%%%%%%%%%%%%%%%%%%%%%%%%%%%%%%%%
%%%%%%%%%%%%%%%%%%%%%%%%%%%%%%%%%%%%%%%%%%%%%%%%%%%%%%%%%%%%%%%%%%%%%%%%%%%%%%%%%%%%%%%

\begin{center}
    {\Large\bfseries \myTitle} \\[2mm]
    \myName \\
\end{center}

\noindent\rule[-1ex]{\textwidth}{1pt}\\[3ex]

\noindent{\textbf{Palabras clave}: Paparazzi UAV, robótica, firmware, drones, GVF} \\

\vspace{3mm}

\noindent{\textbf{Resumen}} \\

Los recientes avances en electrónica, robótica e informática han permitido el desarrollo y creación de un sinfín de nuevas tecnologías.
Entre ellas, ha destacado recientemente el crecimiento de los drones, formalmente conocidos como UAV (\textit{Unmanned Aerial Vehicle}), 
dispositivos que permiten una gran variedad de aplicaciones, desde juguetes y hobbies, hasta aplicaciones profesionales como la agricultura o fotografía. \\

El abanico de posibilidades es uno de los mayores fuertes de esta tecnología, sin embargo, los resultados obtenidos no siempre son los esperados. 
La mayoría de soluciones sugeridas recientemente se apoyan en incrementar el número de UAVs trabajando simultáneamente, lo que implica que se necesita control y coordinación entre los vehículos, 
generalmente con la ayuda de la informática y las telecomunicaciones. \\

En este trabajo nos centraremos en el control y coordinación de varios UAVs, principalmente en la subcategoría conocida como cuadricópteros, popularmente llamados drones.  
Se propone como objetivo final conseguir la coordinación con drones \textit{Crazyflies}.
La integración de resultados se hará mediante la implementación de un firmware de código abierto basado en el proyecto \textit{Paparazzi UAV} y la inclusión de un algoritmo para movimiento coordinado entre drones.

%%%%%%%%%%%%%%%%%%%%%%%%%%%%%%%%%%%%%%%%%%%%%%%%%%%%%%%%%%%%%%%%%%%%%%%%%%%%%%%%%%%%%%%
%%%%%%%%%%%%%%%%%%%%%%%%%%%%%%%%%%%%%%%%%%%%%%%%%%%%%%%%%%%%%%%%%%%%%%%%%%%%%%%%%%%%%%%
%%%%%%%%%%%%%%%%%%%%%%%%%%%%%%%%%%%%%%%%%%%%%%%%%%%%%%%%%%%%%%%%%%%%%%%%%%%%%%%%%%%%%%%

\chapter*{}

\thispagestyle{empty}

%%%%%%%%%%%%%%%%%%%%%%%%%%%%%%%%%%%%%%%%%%%%%%%%%%%%%%%%%%%%%%%%%%%%%%%%%%%%%%%%%%%%%%%
%%%%%%%%%%%%%%%%%%%%%%%%%%%%%%%%%%%%%%%%%%%%%%%%%%%%%%%%%%%%%%%%%%%%%%%%%%%%%%%%%%%%%%%
%%%%%%%%%%%%%%%%%%%%%%%%%%%%%%%%%%%%%%%%%%%%%%%%%%%%%%%%%%%%%%%%%%%%%%%%%%%%%%%%%%%%%%%

\begin{center}
    {\Large\bfseries \myTitleEnglish} \\[2mm]
    \myName \\
\end{center}

\noindent\rule[-1ex]{\textwidth}{1pt}\\[3ex]

\noindent{\textbf{Keywords}: Paparazzi UAV, robotics, firmware, drones, GVF} \\

\vspace{3mm}

\noindent{\textbf{Abstract}} \\

Recent advancements in electronics, robotics and computing led to the development of countless new technologies. Among them, the
growth of drones, formally known as UAV (Unmanned Aerial Vehicles), has recently skyrocketed. \\

The range of possibilities is one of the greatest strengths of this technology; however, obtained results are not always what is expected from it. 
Recently proposed solutions suggest increasing the number of UAVs working simultaneously, 
which implies the need for control and coordination among vehicles, which is typically assisted by computer science and telecommunications. \\

In this work, we will focus on the control and coordination of multiple UAVs, primarily in the subcategory known as quadcopters, popularly called drones.  
The ultimate goal is to achieve coordination among \textit{Crazyflies}.
In order to achieve this goal, we will use an open-source firmware based on the \textit{Paparazzi UAV} project and the implementation of an algorithm for coordinated movement between drones.

%%%%%%%%%%%%%%%%%%%%%%%%%%%%%%%%%%%%%%%%%%%%%%%%%%%%%%%%%%%%%%%%%%%%%%%%%%%%%%%%%%%%%%%
%%%%%%%%%%%%%%%%%%%%%%%%%%%%%%%%%%%%%%%%%%%%%%%%%%%%%%%%%%%%%%%%%%%%%%%%%%%%%%%%%%%%%%%
%%%%%%%%%%%%%%%%%%%%%%%%%%%%%%%%%%%%%%%%%%%%%%%%%%%%%%%%%%%%%%%%%%%%%%%%%%%%%%%%%%%%%%%

\chapter*{}

\thispagestyle{empty}

\noindent\rule[-2ex]{\textwidth}{1.5pt}\\[3.5ex]

Yo, \textbf{\myName}, alumno de la titulación \myDegree\ de la \textbf{\myFaculty}, con DNI 5*******B, autorizo la
ubicación de la siguiente copia de mi Trabajo Fin de Grado en la biblioteca del centro para que pueda ser
consultada por las personas que lo deseen.

\vspace{6cm}

\noindent \textbf{Firmado:} \myName

\vspace{2cm}

\begin{flushright}
    Granada a 21 de Junio de 2024
\end{flushright}

%%%%%%%%%%%%%%%%%%%%%%%%%%%%%%%%%%%%%%%%%%%%%%%%%%%%%%%%%%%%%%%%%%%%%%%%%%%%%%%%%%%%%%%
%%%%%%%%%%%%%%%%%%%%%%%%%%%%%%%%%%%%%%%%%%%%%%%%%%%%%%%%%%%%%%%%%%%%%%%%%%%%%%%%%%%%%%%
%%%%%%%%%%%%%%%%%%%%%%%%%%%%%%%%%%%%%%%%%%%%%%%%%%%%%%%%%%%%%%%%%%%%%%%%%%%%%%%%%%%%%%%

\chapter*{}

\thispagestyle{empty}

\noindent\rule[-2ex]{\textwidth}{1.5pt}\\[3.5ex]

D. \textbf{\myProf}, Profesor del Departamento \myDepartmentAlt\ de la Universidad de Granada.

\vspace{0.5cm}

% D. \textbf{\myOtherProf}, Profesor del Departamento \myDepartmentAlt\ de la Universidad de Granada.

% \vspace{0.5cm}

% \textbf{Informan:}
\textbf{Informa:}

\vspace{0.5cm}

Que el presente trabajo, titulado \textit{\textbf{\myTitle}},
ha sido realizado bajo su supervisión por \textbf{\myName}, y 
% autorizamos
autorizo
la defensa de dicho trabajo ante el tribunal que corresponda.

\vspace{0.5cm}

Y para que conste, expiden y firman el presente informe en Granada a 21 de Junio de 2024

\vspace{1cm}

% \textbf{Los directores:}
\textbf{El director:}

\vspace{5cm}

\begin{center}
    % \noindent \textbf{\myProf \ \ \ \ \ \ \ \myOtherProf}
    \noindent \centering \textbf{\myProf}
\end{center}

%%%%%%%%%%%%%%%%%%%%%%%%%%%%%%%%%%%%%%%%%%%%%%%%%%%%%%%%%%%%%%%%%%%%%%%%%%%%%%%%%%%%%%%
%%%%%%%%%%%%%%%%%%%%%%%%%%%%%%%%%%%%%%%%%%%%%%%%%%%%%%%%%%%%%%%%%%%%%%%%%%%%%%%%%%%%%%%
%%%%%%%%%%%%%%%%%%%%%%%%%%%%%%%%%%%%%%%%%%%%%%%%%%%%%%%%%%%%%%%%%%%%%%%%%%%%%%%%%%%%%%%

\chapter*{Agradecimientos}

\thispagestyle{empty}

A mis amigos, por las ``noches de durum", ``miércoles de pollo"\
y todos los buenos ratos que hemos pasado juntos, 
ya fuese en el futbolín, jugando a cualquier juego o simplemente charlando en el comedor de la ETSIIT. 
En especial, a José por tantísimas cosas que hemos hecho juntos: 
trabajos, jugar, cocinar, el robot... \\

A mi familia, por el apoyo y paciencia durante estos cuatro años. 
Por un lado, a mi padre, por enseñarme a ``cacharrear", 
a mi hermano, por todo lo que me ha enseñado de tecnología e ingeniería y 
a mi tío Alberto, por hacerme ver la tecnología desde el punto de vista del entretenimiento.
Por otro lado, a mi madre, por enseñarme de tantas cosas distintas a la tecnología que siempre son útiles y por tantos otros consejos; 
y a mi hermana, por contagiarme su simpatía, ánimo y sonrisa que todos los días luce. \\

A los miembros del despacho DB-4.
A Manu, por hacer todos los días en el despacho mucho más divertidos. 
En especial, a mi tutor Héctor y su estudiante Jesús, por todo el apoyo brindado a lo largo de este trabajo, 
por ayudarme a salir de todos esos bloqueos, las sugerencias dadas para este trabajo y 
el extenso conocimiento de matemáticas, física y robótica que me han transmitido.
Y por supuesto, por los entretenidos días de pruebas en el aeródromo; 
seguro que no olvidaremos el día en que un Crazyflie voló a toda mecha hacia el barranco... \\
