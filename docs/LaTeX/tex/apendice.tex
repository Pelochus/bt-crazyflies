\appendix

\chapter{Instalación de Paparazzi en Debian}\label{appendix:debian_install}

Paparazzi UAV solo se puede instalar oficialmente en Ubuntu \cite{paparazzi_center_install} sin realizar pasos complejos como compilación desde cero \cite{paparazzi_scratch_install}.

Desde el equipo de Swarm Systems Lab preferimos el uso de Debian a Ubuntu y, debido a su similitud,
el autor de este TFG ha descubierto un método para la instalación de Paparazzi en Debian 12, más detallado en el repositorio \cite{bt-crazyflies}.

La instalación completa se basa en ejecutar el siguiente comando en Debian. Se recomienda ejecutar el script bajo el directorio donde se desee instalar Paparazzi:

\begin{lstlisting}[style=CodigoBash]
$ wget https://raw.githubusercontent.com/Pelochus/bt-crazyflies/main/scripts/debian/paparazzi-debian-install.sh && bash paparazzi-debian-install.sh
\end{lstlisting}

En las dos siguientes secciones se explica como se instaló en detalle Paparazzi en Debian.

\section{Instalación de Paparazzi Center}

Se ha creado el siguiente script de instalación en Bash para automatizar el proceso:

\begin{lstlisting}[style=CodigoBash]
#!/bin/bash

# Needed to force a PPA repo in Debian
sudo apt install -y build-essential devscripts

# Important note: [trusted=yes] is not recommended
echo "deb [trusted=yes] https://ppa.launchpadcontent.net/paparazzi-uav/ppa/ubuntu jammy main" | sudo tee -a /etc/apt/sources.list
echo "deb-src [trusted=yes] https://ppa.launchpadcontent.net/paparazzi-uav/ppa/ubuntu jammy main" | sudo tee -a /etc/apt/sources.list

# Rest of the official Paparazzi guide
sudo apt update
sudo apt install -y paparazzi-dev gcc-arm-none-eabi gdb-multiarch python-is-python3 paparazzi-jsbsim dfu-util

# Clone Paparazzi
git clone --origin upstream https://github.com/paparazzi/paparazzi.git
cd paparazzi
git checkout v6.3 # Currently last stable version

# Compile latest stable
make
\end{lstlisting}

Para entender este script, hay que hacer hincapié en las siguientes líneas:

\begin{itemize}
    \item \textbf{Líneas 11 y 12:} Añadimos los repositorios de descarga para Paparazzi. Para que funcionen en Debian, 
    se añade \texttt{\textbf{[trusted=yes]}} de forma que se fuerce la confianza en el repositorio, ya que no es posible añadirlos de la forma natural según la guía. 
    Se indica además, que se utilice los paquetes para la última version de Ubuntu que disponen desde el equipo de Paparazzi, en este caso, \texttt{\textbf{jammy}}.

    \item \textbf{Línea 16:} Se instalan todos los paquetes según la guía oficial para Ubuntu, exceptuando el paquete para Paparazzi GCS, que no es posible utilizarlo en Debian
\end{itemize}

El resto de líneas son extraídas de la guía oficial de instalación.

Este script debe funcionar en Debian 12 recién instalado si se posee conexión a la red. 
Es posible, aunque no ha sido comprobado, que funcione en distribuciones basadas en Debian u otras versiones de Debian como Debian 11. 
No se recomienda utilizar en distribuciones basadas en Ubuntu, se puede seguir la guía oficial en estos casos.

\section{Instalación de Paparazzi GCS}
Como se ha comentado en la sección previa, el paquete correspondiente a Paparazzi GCS no se puede instalar bajo Debian.
Para ello, se puede utilizar el \textbf{AppImage} oficial, que funciona similar a un ejecutable en Windows o un contenedor de Docker.

Este AppImage necesita ser integrado con el ejecutable de Paparazzi, ya que por defecto este asume que se ha instalado 
el GCS desde el gestor de paquetes \texttt{apt} y, por tanto, hay acceso universal.

En este caso, se utiliza este script para la configuración e integración de Paparazzi GCS en Debian:

\begin{lstlisting}[style=CodigoBash]
#!/bin/bash

cd paparazzi

# Make Paparazzi GCS AppImage Work
echo "# Needed for Paparazzi AppImage to work" >> /home/$USER/.bashrc
echo "export PAPARAZZI_HOME=$(pwd)" >> /home/$USER/.bashrc
echo "export PAPARAZZI_SRC=$(pwd)" >> /home/$USER/.bashrc

# Get AppImage and move to /usr/bin/pprzgcs so it can be launched by Paparazzi Center
sudo apt install -y wget # Just in case, you should have it anyway
sudo wget https://github.com/paparazzi/PprzGCS/releases/download/v1.0.11/pprzgcs-v1.0.11-x86_64.AppImage -O /usr/bin/pprzgcs
sudo chmod 755 /usr/bin/pprzgcs
\end{lstlisting}

En este caso, es mucho más sencillo entender este script:

\begin{itemize}
    \item Primeramente se establecen las variables de entorno \texttt{PAPARAZZI\_HOME} y 
    \texttt{PAPARAZZI\_SRC}, que son necesarias para el funcionamiento del AppImage.

    \item Utilizando el ejecutable \texttt{wget}, descargamos automáticamente el AppImage y lo movemos a \texttt{/usr/bin/pprzgcs}, para que simule el comportamiento de descargar el paquete oficial de la repo. 
\end{itemize}

En general, la instalación de Paparazzi en Debian se condensa en ejecutar el primer script de este apéndice, 
que se ocupa de llamar a los otros dos scripts, de forma que se pueda modularizar la instalación de ambos componentes

\section{Instalación completa}

Para la instalación completa del código fuente de Paparazzi, 
se ejecuta el siguiente comando en el directorio raíz del repositorio:

\begin{lstlisting}[style=CodigoBash]
$ git submodule update --init --recursive .
\end{lstlisting}

Este paso no es necesario a no ser que se necesite compilar código o ejecutar simulaciones
que necesiten de código o librerías externas. 
No es necesario para el funcionamiento de Paparazzi Center y GCS.

%%%%%%%%%%%%%%%%%%%%%%%%%%%%%%%%%%%%%%%%%%%%%%%%%%%%%%%%%%%%%%%%%%%%%%%%%%%%%%%%%%%%%%%%%%%%%%%%%%%%%%%%%%%%%%%%%
%%%%%%%%%%%%%%%%%%%%%%%%%%%%%%%%%%%%%%%%%%%%%%%%%%%%%%%%%%%%%%%%%%%%%%%%%%%%%%%%%%%%%%%%%%%%%%%%%%%%%%%%%%%%%%%%%
%%%%%%%%%%%%%%%%%%%%%%%%%%%%%%%%%%%%%%%%%%%%%%%%%%%%%%%%%%%%%%%%%%%%%%%%%%%%%%%%%%%%%%%%%%%%%%%%%%%%%%%%%%%%%%%%%

% Apendice B

%%%%%%%%%%%%%%%%%%%%%%%%%%%%%%%%%%%%%%%%%%%%%%%%%%%%%%%%%%%%%%%%%%%%%%%%%%%%%%%%%%%%%%%%%%%%%%%%%%%%%%%%%%%%%%%%%
%%%%%%%%%%%%%%%%%%%%%%%%%%%%%%%%%%%%%%%%%%%%%%%%%%%%%%%%%%%%%%%%%%%%%%%%%%%%%%%%%%%%%%%%%%%%%%%%%%%%%%%%%%%%%%%%%
%%%%%%%%%%%%%%%%%%%%%%%%%%%%%%%%%%%%%%%%%%%%%%%%%%%%%%%%%%%%%%%%%%%%%%%%%%%%%%%%%%%%%%%%%%%%%%%%%%%%%%%%%%%%%%%%%

\chapter{Instalación de dependencias}\label{appendix:crazyflie_dependencies}

En el \autoref{appendix:debian_install} se incluye como se adaptó e instaló Paparazzi para Debian, no obstante,
no se incluye que librerías y configuraciones son necesarias para el uso de Crazyflie en Paparazzi.

En este segundo apartado del apéndice se incluye una breve descripción de las librerías y configuraciones necesarias. 
Se incluye aquí otro script de Bash que permite realizar todo esto de manera rápida:

\begin{lstlisting}[style=CodigoBash]
$ wget https://raw.githubusercontent.com/Pelochus/bt-crazyflies/main/scripts/deps/deps-install.sh && bash deps-install.sh
\end{lstlisting}

\section{Dependencias a instalar}
Se necesitan las siguientes dependencias:

\begin{itemize}
    \item \textbf{dfu-util y dfu-programmer:} 
    Paquetes de Ubuntu/Debian para cargar el firmware en el Crazyflie.

    \item \textbf{cflib:}
    Librería de Python para poder comunicarse con el Crazyradio 2.0 y, por tanto, un Crazyflie.

    \item \textbf{gedit:}
    Se usa por defecto para editar archivos en Paparazzi desde Paparazzi Center, al estar pensado para Ubuntu. 
    Es opcional, pero es recomendable tenerlo instalado en Debian como opción \textit{fallback}.
\end{itemize}

\section{Configuraciones y ajustes}
Por otro lado los ajustes necesarios son los siguientes:

\begin{itemize}
    \item \textbf{Añadir el usuario al grupo \texttt{dialout}:} 
    Esto sirve para poder cargar el firmware en el Crazyflie sin permisos de superusuario. 

    \item \textbf{Añadir permisos para el Crazyradio:}
    Similarmente, se añaden permisos para poder acceder al Crazyradio por USB sin ser \textit{root}. 
    Se sigue la guia oficial \cite{usb_permissions_crazyradio}.

    \item \textbf{Cambiar el editor de texto por defecto de Paparazzi Center:}
    De nuevo, opcional, pero se puede desde \textit{File - Edit settings - Text editor}
\end{itemize}

%%%%%%%%%%%%%%%%%%%%%%%%%%%%%%%%%%%%%%%%%%%%%%%%%%%%%%%%%%%%%%%%%%%%%%%%%%%%%%%%%%%%%%%%%%%%%%%%%%%%%%%%%%%%%%%%%
%%%%%%%%%%%%%%%%%%%%%%%%%%%%%%%%%%%%%%%%%%%%%%%%%%%%%%%%%%%%%%%%%%%%%%%%%%%%%%%%%%%%%%%%%%%%%%%%%%%%%%%%%%%%%%%%%
%%%%%%%%%%%%%%%%%%%%%%%%%%%%%%%%%%%%%%%%%%%%%%%%%%%%%%%%%%%%%%%%%%%%%%%%%%%%%%%%%%%%%%%%%%%%%%%%%%%%%%%%%%%%%%%%%

% Apendice C

%%%%%%%%%%%%%%%%%%%%%%%%%%%%%%%%%%%%%%%%%%%%%%%%%%%%%%%%%%%%%%%%%%%%%%%%%%%%%%%%%%%%%%%%%%%%%%%%%%%%%%%%%%%%%%%%%
%%%%%%%%%%%%%%%%%%%%%%%%%%%%%%%%%%%%%%%%%%%%%%%%%%%%%%%%%%%%%%%%%%%%%%%%%%%%%%%%%%%%%%%%%%%%%%%%%%%%%%%%%%%%%%%%%
%%%%%%%%%%%%%%%%%%%%%%%%%%%%%%%%%%%%%%%%%%%%%%%%%%%%%%%%%%%%%%%%%%%%%%%%%%%%%%%%%%%%%%%%%%%%%%%%%%%%%%%%%%%%%%%%%

\chapter{Manual de usuario}\label{appendix:user_manual}

Este manual de usuario se divide en dos partes: Paparazzi y coordinación con Bitcraze. 
Se asume que ya se puede ejecutar Paparazzi para la primera parte. 
Para la segunda, se asume que se tienen las librerías y aplicaciones de Bitcraze. 
Se puede usar la máquina virtual provista por ellos.

Si esto no es así aún, se recomienda leer los apéndices A y B o seguir las guías oficiales de instalación y configuración, 
tanto de Paparazzi como de Bitcraze, según que se vaya a utilizar.  

\section{Uso de Crazyflies en Paparazzi}

Al usar el repositorio de GitHub del autor de este TFG \cite{bt-crazyflies}, 
si se utiliza el submódulo de Paparazzi, el uso de los Crazyflies se ha simplificado
respecto al explicado en el capítulo 2 y se han añadido los pasos pertinentes para el correcto funcionamiento. Los requisitos previos son:

\begin{itemize}
    \item Se solucionaron algunos errores en el firmware. Esto ya se ha integrado en la repo de Paparazzi oficial.
    \item Se ha de cargar, previo a cargar el firmware de Paparazzi, el \textbf{firmware oficial de septiembre de 2019 (2019.09)}. 
    Si esto no es asi, no habrá correcta comunicación con la radio, ya que el firmware de la radio no es modificado por Paparazzi y este entiende a versiones antiguas.
    \item Se recomienda añadir el Flow Deck V2 mencionado en el capítulo 3, \autoref{fig:flow-deck}. Sin este, los resultados reales serán pésimos.
    \item Por último, se han añadido diversos archivos de configuración XML para facilitar un poco más el uso en Paparazzi
\end{itemize}

Para el uso de un Crazyflie, se tienen los siguientes nuevos pasos respecto al capítulo 2. Se recomienda leer el capítulo 2 para las correspondientes figuras.

\begin{itemize}
    \item En el menú desplegable arriba a la derecha, seleccionar \texttt{conf\_crazyflies.xml}
    \item En el superior izquierda, seleccionar \texttt{crazyflie-2.1} para usar el Crazyflie 2.1. 
    Se tiene soporte experimental para Crazyflie 2.0 añadido por el autor de este TFG, pero no es plenamente funcional.
    \item Para simulación, los Crazyflie no funcionan (fallo de Paparazzi), así que se recomienda seleccionar \texttt{gvf-bebop2-x},
    que tiene el mismo funcionamiento que el Crazyflie (mismos flightplan, telemetría...) pero funciona correctamente en simulación.
    \item En los dos pasos previos, por como se ha diseñado los XML, no se necesita seleccionar otros XMLs. 
    Los flightplan, airframe files, telemetry files y demás configuraciones son las correctas si no se modifican, incluyen calibraciones como las del PID.
    \item En base a si se desea simulación o real, usar uno de los dos previos y compilar \texttt{nps} para simulación o \texttt{ap} para cargar posteriormente al Crazyflie, como en el capítulo 2.
    \item Por último, dirigirse a la pestaña de \texttt{Operation} y lanzar una sesión de \texttt{Simulation} o \texttt{Crazyflie Flight} según cual se desee usar. Con esto, si se cumplen todos los pasos previos y apéndices pertinentes, debería poderse utilizar el GCS para GVF.
\end{itemize}

Finalmente, para coordinación, hay que usar los scripts que se encuentran en
\texttt{sw/ground\_segment/python/gvf} según se desee formaciones circulares o segmentos paralelos. 
Se recomienda \textbf{SÓLO USAR} para simulación, ya que la implementación real no da buenos resultados. 

La ejecución ha de realizarse tras mandar los comandos de GVF a los drones. Una vez estos esten siguiendo una trayectoria, ejecutar el script correspondiente.
Ambos scripts tienen sus menús de ayuda correspondientes para entender mejor como se ejecutan.
También se recomienda ver los ejemplos provistos en la sección 4.3 si los menús de ayuda no son suficientes.

\section{Coordinación de Crazyflies bajo firmware oficial}

Para empezar, se recomienda cargar el último firmware oficial de Bitcraze para utilizar sus herramientas \cite{crazyflie_firmware_upgrade}. 
Además, para el uso como swarm, es necesario asignar una URI distinta a cada Crazyflie mediante la configuración de este en Crazyflie Client. 
Como último requisito, se requiere el Flow Deck v2 y la Crazyradio, 
el primero para correcto movimiento y el segundo para la conexión.

Con esta base, y asumiendo la disponibilidad de las dependencias necesarias 
(estarán todas disponibles si se parte desde Bitcraze VM), 
se pueden utilizar los programas en la carpeta \texttt{python} del repositorio de este TFG \cite{bt-crazyflies}.
Alternativamente, se pueden ver los ejemplos oficiales de Bitcraze \cite{cflib-examples}

La ejecución de todos los programas que se mencionarán ahora será de la siguiente forma:

\begin{lstlisting}[style=CodigoBash]
$ python example.py
\end{lstlisting}

\subsection{Demo}

El archivo es \texttt{crazyflie\_demo.py}

Se recomienda utilizar este primer programa para asegurar el correcto funcionamiento de cada Crazyflie por separado. 
Tan sólo se debe modificar la URI del programa para que coincida con la del Crazyflie que se vaya a usar.

El programa consiste en:

\begin{itemize}
    \item Despegar el Crazyflie a 0.5 metros
    \item Moverse 0.5 metros hacia delante
    \item Volver 0.5 metros hacia atrás, tras girar 180 grados
    \item Aterrizar
\end{itemize}

\subsection{Demo para varios Crazyflies}

El archivo es \texttt{crazyflie\_swarm\_demo.py}

Similarmente al caso anterior, esto prueba que el funcionamiento en enjambre funciona.
Se debe modificar las URIs deseadas con los Crazyflies que vayan a participar.

El programa consiste en:

\begin{itemize}
    \item Parpadear los LEDs de ambos Crazyflie para asegurar que funcionan coordinadamente
    \item Despegar ambos Crazyflie a 1 metro
    \item Esperar 3 segundos
    \item Aterrizar
\end{itemize}

\subsection{Segmento para un solo Crazyflie}

El archivo es \texttt{crazyflie\_single\_segment.py}

Este programa tiene como objetivo hacer el funcionamiento de la coordinación en segmentos
pero para un solo dron, de forma que se pueda verificar que los drones se mueven correctamente a lo largo de un segmento.

Este programa consiste en:

\begin{itemize}
    \item Despegar el Crazyflie a 0.5 metros sobre el (0, 0)
    \item Moverse hasta (0.5, 0), volver hacia atras hasta la posición estimada (-0.5, 0). Las unidades son metros. Esto se hará en bucle infinito.
    \item El aterrizaje se hará tras comandar \texttt{Ctrl+C} al programa
\end{itemize}

\subsection{Coordinación en segmentos paralelos para dos Crazyflies}

El archivo es \texttt{crazyflie\_segment\_formation.py}

Este programa tiene como objetivo hacer el funcionamiento de la coordinación en segmentos que tiene este TFG como objetivo.
Se puede ampliar para funcionar con más de dos drones, pero requiere varias modificaciones. 

Este programa consiste en:

\begin{itemize}
    \item Despegar ambos Crazyflie a 0.5 metros sobre el (0, 0). 
    Cada Crazyflie en este caso, tendrán un (0, 0) distinto, se recomienda poner uno al lado del otro, separados aproximadamente medio metro.
    \item Moverse hasta (1, 0), volver hacia atras hasta la posición estimada (-1, 0). 
    Las unidades son metros. Esto se hará en bucle infinito.
    \item Mientras esto se hace, ambos Crazyflie tratarán de mantenerse uno al lado del otro.
    Por defecto, el script hará a uno de los 2 Crazyflies esperar a que el otro avance, de esta forma, tiene más sentido la coordinación. 
    Se deberá cambiar la variable \texttt{waiting\_uri} para elegir cual esperará.
    \item El aterrizaje se hará tras comandar \texttt{Ctrl+C} al programa
\end{itemize}

Si este programa funciona correctamente, deberán obtenerse resultados similares a los de este TFG.

\subsection{Coordinación en segmentos paralelos simulada}

El archivo es \texttt{sim/experiment.py}

Este programa tiene como objetivo hacer el funcionamiento de la coordinación en segmentos 
que tiene este TFG como objetivo, pero simulando uno de los dos drones.
Dado que este programa es un breve experimento, 
no esta pensado para ser usado sin previas modificaciones, añadidos o trabajos manuales.

Para usar este programa, se recomienda previamente tener un log similar al provisto en \texttt{sim/log.txt},
es decir, una sola columna con números que son las posiciones del dron real.
Alternativamente, para probar el funcionamiento se recomienda probar con el archivo provisto de ejemplo.
Se adjunta un script para convertir las comas a puntos si fuese necesario.

Para conseguir un log así se puede utilizar el programa para un solo segmento del Crazyflie explicado en este apéndice y extraer la columna de posiciones.